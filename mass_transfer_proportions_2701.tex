% Define document class
\documentclass[twocolumn]{aastex63}
\DeclareRobustCommand{\Eqref}[1]{Eq.~\ref{#1}}
\DeclareRobustCommand{\Figref}[1]{Fig.~\ref{#1}}
\DeclareRobustCommand{\Tabref}[1]{Tab.~\ref{#1}}
\DeclareRobustCommand{\Secref}[1]{Sec.~\ref{#1}}
\newcommand{\todo}[1]{{\large $\blacksquare$~\textbf{\color{red}[#1]}}~$\blacksquare$}
\newcommand{\mr}[1]{{\textbf{\color{green!75!black}[#1]}}}
% \usepackage{cuted}
% \usepackage{flushend}
\usepackage{amsmath}
% \graphicspath{{./figures/}}

\begin{document}

% Title
\title{On the prevalence of early mass transfer for very massive binaries}

\author[0009-0008-2061-4946]{C.~A.~Burt}
\affiliation{University of Arizona, Department of Astronomy \& Steward Observatory, 933 N.~Cherry Ave., Tucson, AZ 85721, USA}

\author[0000-0002-6718-9472]{M.~Renzo}
\affiliation{University of Arizona, Department of Astronomy \& Steward Observatory, 933 N.~Cherry Ave., Tucson, AZ 85721, USA}

\author[0000-0002-2215-1841]{A.~Grichener}
\affiliation{University of Arizona, Department of Astronomy \& Steward Observatory, 933 N.~Cherry Ave., Tucson, AZ 85721, USA}

\author{\todo{TBD}}

\begin{abstract}
  Common phases of mass transfer in massive stellar binaries are case A
  (during the donor's main sequence) and case B (after the donor's
  main sequence but before helium core depletion). Most stars see
  stellar radii significantly grow after the main sequence, making
  case B more common. However, very massive stars may already undergo
  significant expansion during the main sequence increasing the
  probability of case A mass transfer. We find that using convective
  boundary mixing informed by the width of the main sequence in 30
  Doradus, cass A mass transfer dominates for donor masses
  $\gtrsim 75 \, M_{\odot}$. This is not the case without convective
  boundary mixing or in the stellar models commonly used in rapid
  binary population synthesis.  Therefore, case A mass transfer may be
  more dominant than commonly assumed, with potential impact on rates
  of all post mass transfer binaries, from Wolf-rayet-O-type binaries,
  to X-ray binaries, and gravitational wave progenitors.
\end{abstract}

\section{Mass Transfer in Very Massive Binaries}

Binary stars with a sufficiently small orbital separation
($a\lesssim2500\,R_{\odot}$, e.g.~\citealt{sana:12}) undergo (at least
one) mass transfer phases in which the donor star transfers mass to
the initially less massive accretor. For very massive stars
($ \gtrsim 30 \, M_{\odot}$), mass transfer most often occurs during
the donor's hydrogen core-burning phase (case A) or helium
core-burning phase (case B), which together account for $\sim99\%$ of
the donor's lifetime \citep{kippenhahn:67}.

\begin{figure*}[htbp]
  \centering
  \includegraphics[width=1.0\textwidth]{radii}
  \caption{Each panel contain 15 stellar \textsc{MESA} models spanning
    from $30 \, M_{\odot}$ star (longer lifetimes) to
    $100 \, M_{\odot}$ (shorter lifetimes). The top panels show models
    with \cite{brott:11}-like convective boundary mixing (``large''
    overshooting) \citep{claret:18}, the middle show models without
    overshooting, and the bottom panels plot models generated from
    \textsc{COMPAS} based on analytic fits to the stellar models of
    \cite{pols:98}. The left (right) panels have a metallicity
    $Z=0.001$ ($Z=0.0001$). Dashed red (blue) lines mark the maximum
    radius during the main sequence (during the runtime of the
    model). The red (blue) shaded areas underneath correspond to case
    A (case B) mass transfer.}
  \label{fig:radii}
\end{figure*}


For a flat in $\log_{10}(a)$ initial separation distribution
\citep{opik:24}, case B is expected to occur more often than case A
mass transfer, since most stars expand dramatically post-main sequence
\citep[e.g.,][]{vandenheuvel:69}. However, very massive stars may
already undergo a drastic expansion in radius during their main
sequence \citep[e.g.,][]{sanyal:15, jiang:15}. This may increase the
rate of case A \citep{demink:08}, affecting the evolution of the star
and consequently potentially altering stellar parameters that
determine the rate of Wolf-Rayet+O-type binaries
\citep[e.g.,][]{nuijten:24}, and X-ray binaries and gravitational wave
progenitors \citep[e.g.,][]{mandel:22}.

The radius of the donor is dependent on poorly constrained stellar
parameters, including stellar winds \citep{renzo:17, josiek:24},
metallicity \citep{xin:22}, treatment of
close-to-super-Eddington-layers \citep[e.g.,][]{joss:73, paxton:13,
  jiang:15, agrawal:22, jermyn:23}, and convective boundary mixing
\citep{anders:23, johnston:24}. Here, we explore the radial evolution
of very massive stars and its impact on when mass transfer starts. We
also show variations in convective boundary mixing and metallicity and
compare them to stellar models commonly adopted in rapid binary
population synthesis.

\section{Comparing Donor Radii}

We computed \textsc{MESA} models\footnote{Publicly available at
  \href{https://doi.org/10.5281/zenodo.14757819}{doi.org/10.5281/zenodo.14757819}}
(version 24.03.1, \citealt{paxton:11, paxton:13, paxton:15, paxton:18,
  paxton:19, jermyn:23}) from $30 \, M_{\odot}$ to $100 \, M_{\odot}$
in steps of $5\,M_\odot$ at metallicity $Z=0.001$ and $0.0001$
following the setup from \cite{renzo:23}. The gray lines in
\Figref{fig:radii} show their radial evolution as a function of time.

When using overshooting (top), our \textsc{MESA} models implement an
exponential algorithm \citep{herwig:00} fit to the step overshooting
calibrated on the width of main sequence in 30 Doradus
\citep[$\sim{}0.335$ pressure scale heights,][]{brott:11} following
\cite{claret:18}, corresponding to
\texttt{(f,f\_0)=(0.0415,0.008)}. We compare this relatively ``large
overshooting'' model with models not including any convective boundary
mixing (middle), and to the \cite{pols:98} models including an
effectively mass-dependent overshooting. These are the stellar models
underpinning \textsc{SSE} \citep{hurley:00}, which we generated using
\textsc{COMPAS} \citep{stevenson:17, vignagomez:18, riley:22}. We note
that stars with masses $\geq50\,M_\odot$ in
\textsc{SSE}/\textsc{COMPAS} are the result of extrapolation of the
fits to the models of \cite{pols:98}.

The red and blue dashed lines in each panel of \Figref{fig:radii}
denote the maximum radius during the main sequence and helium core
burning phase, respectively. These mark the maximum Roche radius for a
case A and case B donor, respectively. The right axis shows orbital
separations $a$ where the stellar radius meets the roche radius
computed from \citet{eggleton:83}, assuming a representative
accretor-to-donor mass ratio of $q=0.55$. This value corresponds to
the average for a flat mass-ratio distribution between 0.1 and 1
\citep{kobulnicky:07,sana:12}. The red regions denote binaries which
will undergo case A mass transfer and the blue regions denote binaries
which will undergo case B mass transfer. For $Z=0.001$, when including
overshooting (top left panel), donors with masses
$ \gtrsim 75 \, M_{\odot}$ overwhelmingly experience case A.  Removing
convective boundary mixing (middle) keeps main sequence radii smaller,
preserving the blue region above the red line for case B mass transfer
at all masses. The overshooting implementation from \cite{pols:98}
(bottom), while nonzero, still leaves a large window for case B up to
at least $100 \, M_{\odot}$. At even lower metallicities of $Z=0.0001$
(right), stars are more compact, and all models allow for case B mass
transfer at all masses.

\section{Implications for Post-Mass-Transfer Binaries}

Convective boundary mixing \citep{brott:11, johnston:24} and
metallicity have a strong effect on stellar radii, which determine
when a donor fills its Roche lobe. Related effects on stellar radii
have been explored elsewhere, including the adopted wind mass loss
rates \citep[e.g.,][]{smith:14, renzo:17, josiek:24}, rotation (and
consequently tides, e.g., \citealt{maeder:00}), and the treatment of
energy transport in correspondende of opacity bumps in the envelope
\citep[e.g.,][]{joss:73, agrawal:22, cheng:24}.

After a thermal-timescale initial phase, Case A mass transfer occurs
overall on a longer (nuclear) timescale, while case B occurs entirely
on a much shorter (thermal) timescale \citep[but see][]{klencki:22}.
Moreover, the dynamical stability of the orbit during mass transfer is
sensitive to the evolutionary phase of the stars involved
\citep[e.g.,][]{claeys:14}. Therefore, whether a given binary
experiences dynamically unstable mass transfer,which is critical to
determine the outcome of the binary system, depends on more than just
the mass ratio, as generally assumed in rapid population
synthesis. Comparing rows in \Figref{fig:radii} shows that the stellar
evolution models commonly used in rapid population synthesis are
qualitatively similar to our no overshooting models, in the sense that
they allow for case B mass transfer in all mass regions we sampled and
for metallicities relevant to galactic and gravitational astronomy.

Given the critical role of mass transfer for the formation of many
binaries of interest, the fraction of systems experiencing case A in
respect to case B may significany impact predicted rates for post mass
transfer binaries, including Wolf-Rayet+O-type binaries, X-ray
generated usingbinaries, and gravitational wave progenitors. In
particular, the role of the stable mass transfer channel
\citep[e.g.,][]{marchant:21, vanson:22} for (massive) binary black
hole mergers is currently hotly debated. Our results highlight that
stellar uncertainties influence the mode of mass transfer and
consequently the outcomes.


\software{This work made use of the following software packages:
  \texttt{matplotlib} \citep{Hunter:2007} and \texttt{python}
  \citep{python}, \url{http://github.com/TeamCOMPAS/COMPAS}, and
  \url{https://docs.mesastar.org} MESA. Software citation information
  aggregated using
  \texttt{\href{https://www.tomwagg.com/software-citation-station/}{The
      Software Citation Station}}
  \citep{software-citation-station-paper,
    software-citation-station-zenodo}.}

\bibliography{./donorR.bib}
\end{document}

%%% Local Variables:
%%% mode: LaTeX
%%% TeX-master: t
%%% End:
