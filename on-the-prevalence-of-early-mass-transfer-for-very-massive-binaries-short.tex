% Define document class
\documentclass[twocolumn]{aastex631}
\DeclareRobustCommand{\Eqref}[1]{Eq.~\ref{#1}}
\DeclareRobustCommand{\Figref}[1]{Fig.~\ref{#1}}
\DeclareRobustCommand{\Tabref}[1]{Tab.~\ref{#1}}
\DeclareRobustCommand{\Secref}[1]{Sec.~\ref{#1}}
\newcommand{\todo}[1]{{\large $\blacksquare$~\textbf{\color{red}[#1]}}~$\blacksquare$}
\newcommand{\mr}[1]{{\textbf{\color{green!75!black}[#1]}}}
% \usepackage{cuted}
% \usepackage{flushend}
\usepackage{amsmath}
% \graphicspath{{./figures/}}

\begin{document}

% Title
\title{On the prevalence of early mass transfer for very massive binaries}

\author[0009-0008-2061-4946]{C.~A.~Burt}
\affiliation{University of Arizona, Department of Astronomy \& Steward Observatory, 933 N.~Cherry Ave., Tucson, AZ 85721, USA}

\author[0000-0002-6718-9472]{M.~Renzo}
\affiliation{University of Arizona, Department of Astronomy \& Steward Observatory, 933 N.~Cherry Ave., Tucson, AZ 85721, USA}

\author[0000-0002-2215-1841]{A.~Grichener}
\affiliation{University of Arizona, Department of Astronomy \& Steward Observatory, 933 N.~Cherry Ave., Tucson, AZ 85721, USA}

\author[0000-0002-8465-8090]{N.~Shah}
\affiliation{University of Arizona, Department of Astronomy \& Steward Observatory, 933 N.~Cherry Ave., Tucson, AZ 85721, USA}

\correspondingauthor{C.~A.~Burt}
\email{caburt@arizona.edu}

\correspondingauthor{M.~Renzo}
\email{mrenzo@arizona.edu}


\begin{abstract}
  Common phases of mass transfer in massive stellar binaries are case
  A (during the donor's main sequence) and case B (after the donor's
  main sequence but before helium core depletion). Most stars see
  their radii significantly grow after the main sequence, making case
  B more common. However, very massive stars may already undergo
  significant expansion during the main sequence increasing the
  probability of case A mass transfer. We find that using convective
  boundary mixing informed by the width of the main sequence in 30
  Doradus, case A mass transfer dominates for donor masses
  $\gtrsim 75 \, M_{\odot}$. This is not the case without convective
  boundary mixing or in the stellar models commonly used in rapid
  binary population synthesis. Therefore, case A mass transfer may be
  more dominant than commonly assumed, with potential impact on rates
  of all post mass transfer binaries, from Wolf-Rayet-O-type binaries,
  to X-ray binaries and gravitational wave progenitors.
\end{abstract}

\section{Mass Transfer in Very Massive Binaries}

Binaries with orbital separation $a\lesssim2500\,R_{\odot}$ undergo
(at least one) mass transfer phases \citep{sana:12}. For very massive
donors ($ \gtrsim 30 \, M_{\odot}$), mass transfer most often occurs
during the donor's hydrogen (case A) or helium core-burning phase
(case B), together accounting for $\sim99\%$ of the lifetime.

\begin{figure*}[htbp]
  \centering
  \includegraphics[width=1.0\textwidth]{radii}
  \caption{Each panel contains 15 stellar \textsc{MESA} models
    spanning from $30 \, M_{\odot}$ star (longer lifetimes) to
    $100 \, M_{\odot}$ (shorter lifetimes). The top panels show models
    with \cite{brott:11}-like ``large'' overshooting
    \citep{claret:18}, the middle panels show models without
    overshooting, and the bottom panels plot models generated using
    \textsc{COMPAS} based on analytic fits to stellar models from
    \cite{pols:98}. The left (right) panels have metallicity $Z=0.001$
    ($Z=0.0001$). Dashed red (blue) lines mark the maximum radius
    during the main sequence (during the model's runtime). The red
    (blue) shaded areas correspond to case A (case B) mass transfer.}
  \label{fig:radii}
\end{figure*}

For a flat-in-$\log_{10}(a)$ initial distribution \citep{opik:24},
case B is expected to be more common, since most stars greatly expand
post-main sequence \citep[e.g.,][]{vandenheuvel:69}. However, very
massive stars may already drastically expand during their main
sequence \citep[e.g.,][]{sanyal:15, jiang:15}. This may increase the
rate of case A \citep{demink:08}, potentially affecting
Wolf-Rayet+O-type binaries \citep[e.g.,][]{nuijten:24}, and X-ray
binaries and gravitational wave progenitors
\citep[e.g.,][]{mandel:22}.

The radius of the donor depends on poorly constrained stellar
parameters, including winds \citep{renzo:17, josiek:24}, metallicity
\citep{xin:22}, treatment of close-to-super-Eddington layers
\citep[e.g.,][]{joss:73, jiang:15} rotation (e.g.,
\citealt{maeder:00}) and consequently tides (e.g., \citealt{fabry:22})
and convective boundary mixing \citep{anders:23, johnston:24}. Here,
we explore the radial evolution of very massive stars and its impact
on when mass transfer starts and compare to models commonly adopted in
rapid binary population synthesis.

\section{Comparing Donor Radii}

We computed \textsc{MESA} models\footnote{Available at
  \href{https://doi.org/10.5281/zenodo.14757819}{doi.org/10.5281/zenodo.14757819}}
(version 24.03.1, \citealt{paxton:11, paxton:13, paxton:15, paxton:18,
  paxton:19, jermyn:23}) from $30 \, M_{\odot}$ to $100 \, M_{\odot}$
in steps of $5\,M_\odot$ at metallicity $Z=0.001$ and $0.0001$
following the setup from \cite{renzo:23}. The gray lines in
\Figref{fig:radii} show their radial evolution as a function of time.

When using overshooting (top), we adopt an exponential algorithm
\citep{herwig:00} fit to the step overshooting calibrated on the width
of main sequence in 30 Doradus \citep[$\sim{}0.335$ pressure scale
heights,][]{brott:11} following \cite{claret:18}, corresponding to
\texttt{(f,f\_0)=(0.0415,0.008)}. We also compute models without any
convective boundary mixing (middle), and show the \cite{pols:98}
models including an effectively mass-dependent overshooting (bottom).
These are the stellar models underpinning \textsc{SSE}
\citep{hurley:00}, which we generated using \textsc{COMPAS}
\citep{stevenson:17, vignagomez:18, riley:22}. Models with
$M\geq50\,M_\odot$ in the bottom panel are extrapolations.

The red and blue dashed lines in each panel of \Figref{fig:radii}
denote the maximum main sequence radius and the maximum radius before
helium depletion, marking the maximum Roche radius for a case A and
case B donor, respectively. The right axis shows $a$ for which the
star fills its Roche lobe \citet{eggleton:83}, assuming a
representative accretor-to-donor mass ratio $q=0.55$. For $Z=0.001$,
when including overshooting (top left panel), donors with masses
$ \gtrsim 75 \, M_{\odot}$ are more likely to experience case
A. Removing convective boundary mixing reduces the stellar radii
during the main sequence, preserving a window for case B. The
overshooting implementation from \cite{pols:98}, while nonzero, still
leaves a large window for case B up to at least $100 \, M_{\odot}$. At
$Z=0.0001$ (right), stars are more compact, and all models allow case
B mass transfer at all masses.

\section{Implications for Post-Mass-Transfer Binaries}

After a thermal-timescale initial phase, Case A mass transfer occurs
overall on a longer (nuclear) timescale, while case B occurs entirely
on a much shorter (thermal) timescale \citep[but see][]{klencki:22}.
Moreover, the dynamical stability of mass transfer is sensitive to the
evolutionary phase \citep[e.g.,][]{claeys:14}. Therefore, whether a
binary experiences dynamically unstable interactions depends on
stellar radii and their reaction to mass changes, both in turn
influenced by stellar physics assumptions.

Given the critical role of mass transfer, the fraction of systems
experiencing case A in respect to case B significantly impacts
predicted rates for post mass transfer binaries. In particular, the
role of the stable mass transfer channel \citep[e.g.,][]{marchant:21,
  vanson:22} for (massive) binary black hole mergers is an active
topic of discussion. Our results highlight that stellar uncertainties
influence the mode of mass transfer and consequently the outcomes.


\software{This work made use of the following software packages:
  \texttt{matplotlib} \citep{Hunter:2007} and \texttt{python}
  \citep{python}, \url{http://github.com/TeamCOMPAS/COMPAS}, and
  \url{https://docs.mesastar.org} MESA. Software citation information
  aggregated using
  \texttt{\href{https://www.tomwagg.com/software-citation-station/}{The
      Software Citation Station}}
  \citep{software-citation-station-paper,
    software-citation-station-zenodo}.}

\bibliography{./donorR-short.bib}
\end{document}

%%% Local Variables:
%%% mode: LaTeX
%%% TeX-master: t
%%% End:
