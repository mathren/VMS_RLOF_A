% Define document class
\documentclass[twocolumn]{aastex63}
\DeclareRobustCommand{\Eqref}[1]{Eq.~\ref{#1}}
\DeclareRobustCommand{\Figref}[1]{Fig.~\ref{#1}}
\DeclareRobustCommand{\Tabref}[1]{Tab.~\ref{#1}}
\DeclareRobustCommand{\Secref}[1]{Sec.~\ref{#1}}
\newcommand{\todo}[1]{{\large $\blacksquare$~\textbf{\color{red}[#1]}}~$\blacksquare$}
\newcommand{\mr}[1]{{\textbf{\color{green!75!black}[#1]}}}
% \usepackage{cuted}
\usepackage{flushend}
\usepackage{amsmath}
\graphicspath{{./figures/}}

\begin{document}

% Title
\title{\todo{title}}

\author{C.~A.~Burt}
\affiliation{University of Arizona, Department of Astronomy \& Steward Observatory, 933 N.~Cherry Ave., Tucson, AZ 85721, USA}

\author[0000-0002-6718-9472]{M.~Renzo}
\affiliation{University of Arizona, Department of Astronomy \& Steward Observatory, 933 N.~Cherry Ave., Tucson, AZ 85721, USA}

\author[0000-0002-2215-1841]{A.~Grichener}
\affiliation{University of Arizona, Department of Astronomy \& Steward Observatory, 933 N.~Cherry Ave., Tucson, AZ 85721, USA}

\author{\todo{TBD}}

\begin{abstract}
  \todo{TBD} \todo{Christian, make an ORCiD account and add the number
    in squared bracket to your author. Decide now if you want to
    publish as C.~Burt or C.~A.~Burt (you should then stick with it
    for the rest of your career)}
\end{abstract}

\section{Mass Transfer Between \mr{Very Massive stars \sout{Binary Black Hole Progenitors}}}

Binary stars with a sufficiently small \mr{orbital separation
  \sout{radius}} undergo a mass transfer phase in which one donor star
transfers mass to an accretor star. For massive stars
($>30M_{\odot}$), this occurs in two cases \mr{this is not correct, to
  most people a 3Msun star is massive and could also have case C}.
Case A mass transfer occurs while the donor star is in the main
sequence\mr{, while c}ase B mass transfer occurs while the donor star
is burning helium in its core \mr{\cite{kippenhahn:67}}. In both cases,
the radius of the \mr{which star?} star expands, eventually
overflowing its Roche Lobe, allowing for the surface to become
gravitationally bound to the accretor \mr{logical order: maybe first
  describe the radius expansion, then the filling of the Roche lobe,
  then introduce case A/B}. The accretor gains the mass lost by the
donor until the masses are equal \mr{no, typically until m2=1.1*(m2
  before RLOF), see \cite{packet:81}}, or the mass gainer cannot
accrete due to increased rotational velocity caused by angular
momentum conservation. The \mr{increase in total mass of the accretor,
  typically a main sequence star during RLOF, causes by the virial
  theorem an increase in its central temperature and thus in the
  extent of core-convection resulting in access to more nuclear fuel
  and \sout{accreted mass has been observed to} elongation of the
  lifespan of the star (\citealt{neo:77} but see also
  \citealt{braun:95}) and potentially modifying its internal structure
  and future evolution \citep[e.g.,]{renzo:23, , wagg:24, landri:24,
    nathaniel:24}.}

\mr{slow down a bit and reorder: first state that typically case B is the most
  common, because stars expand most post-main sequence in the
  Herzsprung gap \citep{vandenheuvel:69} then say something like
  ``however the more massive the star, the greater radial expansion on
  the main sequence'' \citep[e.g.,][]{brott:11} resulting in a change in
  the ratio of number of systems going through case A vs. case B with
  initial donor mass \citep{demink:08}. Please rephrase the paragraph
  below}

\mr{probably already from the ``introduction'' we need to say why we
  care, why would readers care? binary interactions are crucial to the
formation of GW sources, and in particular for BBH stable mass
transfer may dominate \cite{marchant:21, vanson:21}}

The relative proportion of case A mass transfer in comparison to case
B mass transfer increases with mass of the donor star. This is
expected, as larger stars see greater rates of radial expansion during
the main sequence. This has a significant side effect for stars with
($\gtrsim 75M_{\odot}$), case A mass transfer occurs in all possible
mass transfer processes. For stars of lesser mass, the thermal
expansion of the star at the end of the main sequence expands the star
beyond its maximum radius during the main sequence. However, for stars
in this mass range, the radius expands drastically during the main
sequence, such that the maximum radius during the main sequence and
helium core burning phase are similar.

\mr{maybe here you need to explain a bit what we do in this research
  note before jumping in the results!}
\todo{Describe the models you run (which input physics, which code)
  with and without overshooting, and the pols+98 models (just reference)}

\mr{here you want to make sure to not confuse the readers, so the
  paragraph above should be phrased as the typical expectation,
  however, stellar physics uncertainties matter} Convective
\mr{boundary mixing\sout{overshooting}} has a strong effect on stellar
radius \cite{brott:11, johnston:24}, which determines Roche lobe
overflow. \mr{Therefore \sout{Ergo}}, different overshooting models
have a great effect on the nature of mass transfer. The model in the
[panel] that limits case B for stars with mass ($\gtrsim 75M_{\odot}$)
includes exponential overshooting with constants []. In contrast, the
model in the [panel] shows no overshooting and consequently models a
star that is much more compact. As a result, a system with no
overshooting would allow for case B mass transfer to be observed in
all mass ranges


\begin{figure*}[htbp]
  \centering
  % \includegraphics[width=0.9\textwidth]{comparison}
  \caption{\todo{TBD}}
  \label{fig:R_t_donor}
\end{figure*}

\mr{Can you put some bullet points for what you think should go after
  this? Some ideas: explain why we plot also Pols+98 models and
  connect to GWs should be the aim here}


\bibliography{./donorR.bib}
\end{document}

%%% Local Variables:
%%% mode: latex
%%% TeX-master: t
%%% End:
